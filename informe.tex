\documentclass[runningheads]{llncs}
%
% Supongo que lo vamos a escribir en español
%
\usepackage[spanish]{babel}
%
\usepackage[T1]{fontenc}
%
\usepackage{graphicx}
%
\begin{document}
%
\title{Investigación Dafny}
%
\author{Gonzalo Garcia Zurlo\inst{1} \and
Santiago Monserrat Campanello\inst{1} \and
Federico Virgolini\inst{1}}
%
\authorrunning{F. Author et al.}
%
\institute{Universidad Nacional de Córdoba, \\
Facultad de Matemática, Astronomía, Física y Computación, \\
Av. Medina Allende, Córdoba, Argentina
\email{\{gonzalo.garcia.zurlo,smonserratc,federico.virgolini\}@mi.unc.edu.ar}}
%
\maketitle
% inserta el índice general
\begin{abstract}
    The abstract should briefly summarize the contents of the paper in
    150--250 words.
\keywords{Dafny  \and Second keyword \and Another keyword.}
\end{abstract}
%
\tableofcontents
% Esta sección no se si deberia volar o no porque no se que pondriamos
\section{Introducción}
Y bien, aquí comienza mi articulillo.
\section{Contexto de creación de la herramienta}
Aparecio por primera vez en 2009.

Dafny was created by Rustan Leino at Microsoft Research
after his previous work on developing ESC/Modula-3, ESC/Java, and Spec\#.

Creo que vivia en el pacific northwest de estados unidos en esa epoca

La herramienta está super "viva" tiene contribuciones constantemente, la ultima release estable es 4.6.0 salio el 28 de marzo de 2024. Hay commits casi todos los dias

http://leino-online.com/rustan/ (xd tremenda página tiene)

http://leino-online.com/music/B-student.html (tiene canciones epicas fadasvfsjvklfa, escitas el mismo mes del 9/11)

\section{Objetivo de la herramienta}
Dafny was designed as a verification-aware programming language, requiring verification along with code development. It thus fits the "Correct by Construction" software development paradigm.

.

\section{Descripción de la herramienta del lado del usuario}
Como cualquier otro lenguaje de programación más o menos moderno tiene un LSP (Language Server Protocol).

Es un lenguaje de programación con una sintaxis relativamente estandar con las siguentes features:

mathematical and bounded integers and reals, bit-vectors, classes, iterators, arrays, tuples, generic types, refinement and inheritance,
inductive datatypes that can have methods and are suitable for pattern matching,
lazily unbounded datatypes,
subset types, such as for bounded integers,
lambda expressions and functional programming idioms,
and immutable and mutable data structures

Y al lenguaje le agrega esto para verificar: (Dafny also offers an extensive toolbox for mathematical proofs about software, including
)

bounded and unbounded quantifiers,
calculational proofs and the ability to use and prove lemmas,
pre- and post-conditions, termination conditions, loop invariants, and read/write specifications.

Luego de hacer un programa y chequearlo dafny puede compilar (quizas mejor dicho transpilar) el codigo a C\#, Go, Python, Java, or JavaScript (more to come!)

\section{Aspectos técnicos de la herramienta}
The general proof framework is that of Hoare logic.

Dafny builds on the Boogie intermediate language
which uses the Z3 automated theorem prover
(otra herramienta de microsoft reserch https://github.com/Z3Prover/z3) for discharging proof obligations.

\section{Casos de estudio (exitosos o no de la herramienta)}
Parace que se la usa mucho para competencias de specificacion y verificación formal de programas.

Parece que más que nada se lo usa para enseñar verificación (segun la página de microsoft reaserch).

https://www.microsoft.com/en-us/research/uploads/prod/2008/12/dafny\_krml203.pdf

En ese paper hay un caso de estudio del algoritmo schorr-waite (ni idea que es eso)

\section{Comparación con otras herramientas}
https://github.com/backtracking/program-proofs-with-why3

Acá parece que compara dafny con why3 que parece ser otra herramienta que tambien tiene un lenguaje de programación

Estos son algunos problemas que salen en el libro del autor de Dafny

\section{Caso de estudio elegido}
???

\section{Conclusiones particulares}
Es facil de usar y amigable (todavia no escribi nada en Dafny)

\section{Links varios y otras frutas}
https://www.microsoft.com/en-us/research/project/dafny-a-language-and-program-verifier-for-functional-correctness/

La pagina original o eso parece de microsoft research del lenguaje. Tiene buenos resumenes de que se trata la cosa

https://www.microsoft.com/en-us/research/uploads/prod/2008/12/Dafny\_krml190.pdf/

Parece ser otro paper del autor apenas saco Dafny

https://www.linkedin.com/in/k-rustan-m-leino-91a9a213/

ahora el chabon que creo Dafny trabaja para amazon

saco un libro para enseñar pruebas https://program-proofs.com/ , no se que onda estára el libro

tiene un canal de youtube explicando algunas cosas de dafny y verificación en general (https://www.youtube.com/watch?v=spcfzbisBv4 video de ternas de Hoare)
\end{document}