\documentclass[runningheads]{llncs}
%
% Supongo que lo vamos a escribir en español
%
\usepackage[spanish]{babel}
%
\usepackage[T1]{fontenc}
%
\usepackage{graphicx}
%
\begin{document}
%
\title{Investigación Dafny}
%
\author{Gonzalo Garcia Zurlo\inst{1} \and
Santiago Monserrat Campanello\inst{1} \and
Federico Virgolini\inst{1}}
%
\authorrunning{F. Author et al.}
%
\institute{Universidad Nacional de Córdoba, \\
Facultad de Matemática, Astronomía, Física y Computación, \\
Av. Medina Allende, Córdoba, Argentina
\email{\{gonzalo.garcia.zurlo,smonserratc,federico.virgolini\}@mi.unc.edu.ar}}
%
\maketitle
% inserta el índice general
\begin{abstract}
    The abstract should briefly summarize the contents of the paper in
    150--250 words.
\keywords{Dafny  \and Second keyword \and Another keyword.}
\end{abstract}
%
\tableofcontents
% Esta sección no se si deberia volar o no porque no se que pondriamos
\section{Introducción}
Y bien, aquí comienza mi articulillo.
\section{Contexto de creación de la herramienta}

La herramienta Dafny fue creada en 2009 por Rustan Leino,
a partir de experiencias previas desarrollando herramientas como ESC/Modula-3, ESC/java y Spect\#.

Desde 2001, Rustan vivio en pacific northwest (EE. UU.) y sé desempeñó como investigador y desarrollador de software para Microsoft,
donde finalmente creó Dafny.

Previamente, trabajó en el desarrollo del lenguaje de verificación "Boogie",
que hoy en día es un componente básico para muchos verificadores de programas modernos,
como Joogie, GPUVerify, SMACK, VCC o el mismo Dafny.    

Dafny fue creado principalmente en un contexto académico, 
aunque se sigue siendo usado para la enseñanza en muchas universidades, 
Dafny tiene un fuerte enfoque en aplicaciones prácticas con mucha relevancia industrial actualmente.
Un ejemplo de esto es el uso de Dafny en Amazon Web Services (AWS) para verificar la corrección de los servicios de AWS.

https://www.amazon.science/working-at-amazon/rustan-leino-provides-proof-that-software-is-bug-free

\section{Objetivo de la herramienta}
Dafny was designed as a verification-aware programming language, requiring verification along with code development. It thus fits the "Correct by Construction" software development paradigm.

.

\section{Descripción de la herramienta del lado del usuario}
Como cualquier otro lenguaje de programación más o menos moderno tiene un LSP (Language Server Protocol).

Es un lenguaje de programación con una sintaxis relativamente estandar con las siguentes features:
En cuanto a las estructuras de datos a nivel de usuario que maneja, podemos utilizar arreglos, secuencias, conjuntos, mapas y tuplas.

mathematical and bounded integers and reals, bit-vectors, classes, iterators, arrays, tuples, generic types, refinement and inheritance,
inductive datatypes that can have methods and are suitable for pattern matching,
lazily unbounded datatypes,
subset types, such as for bounded integers,
lambda expressions and functional programming idioms,
and immutable and mutable data structures

Y al lenguaje le agrega esto para verificar: (Dafny also offers an extensive toolbox for mathematical proofs about software, including
)

bounded and unbounded quantifiers,
calculational proofs and the ability to use and prove lemmas,
pre- and post-conditions, termination conditions, loop invariants, and read/write specifications.

Luego de hacer un programa y chequearlo dafny puede compilar (quizas mejor dicho transpilar) el codigo a C\#, Go, Python, Java, or JavaScript (more to come!)

\section{Aspectos técnicos de la herramienta}
El marco general en el que se basa Dafny, al igual que muchas herramientas de verificación formal, es el de la lógica de Hoare,
utilizando principios como tripletes de Hoare, reglas de inferencia y correcion parcial y total.

En cuanto al espacio de estado, Dafny maneja este a partir de herramientas como SMT Solvers y Boogie Intermediate Verification Language
para realizar la verificación formal de programas.

En conjunto con esto, Dafny se centra en considerar todas las posibles ejecuciones del programa dentro del marco de las especificaciones proporcionadas. 
Sin embargo, utiliza técnicas que le permiten manejar el espacio de estado de manera simbólica en lugar de explícita, 
lo que le permite considerar efectivamente "todo el espacio de estado" sin necesidad de enumerarlo completamente.

En lugar de trabajar con valores concretos, Dafny utiliza variables simbólicas que representan un rango de posibles valores,
esto permite el análisis de todos los posibles estados del programa simultáneamente.
Además, Dafny utiliza invariantes dentro de bucles, como también precondiciones (requires) y postcondiciones (ensures) para definir las condiciones 
que deben cumplirse antes y después de la ejecución de métodos.

Para simplificar y transformar los problemas de verificación en formas que pueden ser manejadas más eficientemente por los solvers, 
Dafny utiliza técnicas de reducción como reducción de expresiones, desdoblamiento de bucles, eliminación de cuantificadores,
abstracción de datos (para evitar detalles internos) y descomposición modular (para dividir problemas grandes en problemas más pequeños).

En cuanto a la validez de los resultados, la verificación que realiza Dafny es correcta en términos de asegurar que las condiciones
especificadas (precondiciones, postcondiciones, invariantes) se cumplen en todas las ejecuciones posibles del programa. 
Además, dentro del marco de su modelo simbólico y las especificaciones proporcionadas,
Dafny busca ser exhaustivo. Las técnicas de reducción y abstracción antes mencionadas no comprometen la exhaustividad lógica.

Aunque Dafny esté diseñado para ser muy preciso en la verificación formal, 
los falsos negativos pueden ser más comunes de lo esperable debido a las limitaciones prácticas que tiene SMT solver
cuando la complejidad de las especificaciones es grande.

Finalmente, Dafny utiliza varias estructuras de datos y conceptos avanzados para realizar la verificación formal del código,
entre las que podemos encontrar árboles de sintaxis abstracta (ASTs), Boogie Intermediate Representation (IR), grafos de control de flujo (CFGs),
tablas de símbolos y condiciones de verificación (VCs).

\section{Casos de estudio (exitosos o no de la herramienta)}
Parace que se la usa mucho para competencias de specificacion y verificación formal de programas.

Parece que más que nada se lo usa para enseñar verificación (segun la página de microsoft reaserch).

https://www.microsoft.com/en-us/research/uploads/prod/2008/12/dafny\_krml203.pdf

En ese paper hay un caso de estudio del algoritmo schorr-waite (ni idea que es eso)

\section{Comparación con otras herramientas}

Existen varias herramientas de verificación deductiva de programas que utilizan su propio lenguaje de programación 
y que son similares a Dafny en su enfoque de verificación formal, 
un ejemplo de esto es Why3 que utiliza un enfoque basado en condiciones de verificación y SMT solvers.

Otro claro ejemplo es Spec\# que se puede considerar el predeseor de Dafny y que también utiliza un enfoque basado en lógica de Hoare,
y comparten muchos principios y tecnologías subyacentes, como la integración con el SMT solver Z3.

En contra parte, Dafny difiere con otras herramientas de verificacion formal en ciertos aspectos, uno de los ejemplos mas claros es Coq 
que su principal caracteristica es ser interactivo, 
permitiendo a los usuarios construir pruebas paso a paso, mientras que Dafny automatiza la mayor parte del proceso de verificación,
ademas, Coq soporta logicas de ordenes superiores, lo que permite especificaciones y verificaciones más expresivas y complejas que Dafny.

Otro ejemplo puede ser Agda, que difiere con las antes mencionadas ya que 
es orientada hacia la programación funcional, mientras que Dafny es un lenguaje de programación imperativo. 
Tambien como Coq, Agda utiliza tipos dependientes para especificar y verificar propiedades, 
permitiendo una integración más estrecha entre programación y pruebas, mientras que Dafny utiliza tipos basicos y compuestos del paradigma imperativo.

\section{Caso de estudio elegido}
???

\section{Conclusiones particulares}
Es facil de usar (como yo para ella Sad) y amigable (todavia no escribi nada en Dafny)

\section{Links varios y otras frutas}
(otra herramienta de microsoft reserch https://github.com/Z3Prover/z3) for discharging proof obligations.

https://www.microsoft.com/en-us/research/project/dafny-a-language-and-program-verifier-for-functional-correctness/

La pagina original o eso parece de microsoft research del lenguaje. Tiene buenos resumenes de que se trata la cosa

https://www.microsoft.com/en-us/research/uploads/prod/2008/12/Dafny\_krml190.pdf/

Parece ser otro paper del autor apenas saco Dafny

https://www.linkedin.com/in/k-rustan-m-leino-91a9a213/

ahora el chabon que creo Dafny trabaja para amazon

saco un libro para enseñar pruebas https://program-proofs.com/ , no se que onda estára el libro

tiene un canal de youtube explicando algunas cosas de dafny y verificación en general (https://www.youtube.com/watch?v=spcfzbisBv4 video de ternas de Hoare)

https://github.com/backtracking/program-proofs-with-why3

https://en.wikipedia.org/wiki/Agda_(programming_language)

https://en.wikipedia.org/wiki/Coq_(software)

\end{document}

